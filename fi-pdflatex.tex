%%%%%%%%%%%%%%%%%%%%%%%%%%%%%%%%%%%%%%%%%%%%%%%%%%%%%%%%%%%%%%%%%%%%
%% I, the copyright holder of this work, release this work into the
%% public domain. This applies worldwide. In some countries this may
%% not be legally possible; if so: I grant anyone the right to use
%% this work for any purpose, without any conditions, unless such
%% conditions are required by law.
%%%%%%%%%%%%%%%%%%%%%%%%%%%%%%%%%%%%%%%%%%%%%%%%%%%%%%%%%%%%%%%%%%%%

\documentclass[
  digital, %% This option enables the default options for the
           %% digital version of a document. Replace with `printed`
           %% to enable the default options for the printed version
           %% of a document.
  table,   %% Causes the coloring of tables. Replace with `notable`
           %% to restore plain tables.
  lof,     %% Prints the List of Figures. Replace with `nolof` to
           %% hide the List of Figures.
  lot,     %% Prints the List of Tables. Replace with `nolot` to
           %% hide the List of Tables.
  %% More options are listed in the user guide at
  %% <http://mirrors.ctan.org/macros/latex/contrib/fithesis/guide/mu/fi.pdf>.
]{fithesis3}
%% The following section sets up the locales used in the thesis.
\usepackage[resetfonts]{cmap} %% We need to load the T2A font encoding
\usepackage[T1,T2A]{fontenc}  %% to use the Cyrillic fonts with Russian texts.
\usepackage[
  main=english, %% By using `czech` or `slovak` as the main locale
                %% instead of `english`, you can typeset the thesis
                %% in either Czech or Slovak, respectively.
  english, german, russian, czech, slovak %% The additional keys allow
]{babel}        %% foreign texts to be typeset as follows:
%%
%%   \begin{otherlanguage}{german}  ... \end{otherlanguage}
%%   \begin{otherlanguage}{russian} ... \end{otherlanguage}
%%   \begin{otherlanguage}{czech}   ... \end{otherlanguage}
%%   \begin{otherlanguage}{slovak}  ... \end{otherlanguage}
%%
%% For non-Latin scripts, it may be necessary to load additional
%% fonts:
\usepackage{paratype}
\def\textrussian#1{{\usefont{T2A}{PTSerif-TLF}{m}{rm}#1}}
%%
%% The following section sets up the metadata of the thesis.
\thesissetup{
    date          = 2018/05/20,
    university    = mu,
    faculty       = fi,
    type          = mgr,
    author        = Jiří Mauritz,
    gender        = m,
    advisor       = Libor Kubečka,
    title         = {Automatic Categorization of Legal Documents},
    TeXtitle      = {Automatic Categorization of Legal Documents},
    keywords      = {machine learning, legal documents, natural language processing},
    TeXkeywords   = {machine learning, legal documents, natural language processing},
    abstract      = {This is the abstract of my thesis, which can

                     span multiple paragraphs.},
    thanks        = {These are the acknowledgements for my thesis, which can

                     span multiple paragraphs.},
    bib           = example.bib,
}
\usepackage{makeidx}      %% The `makeidx` package contains
\makeindex                %% helper commands for index typesetting.
%% These additional packages are used within the document:
\usepackage{paralist} %% Compact list environments
\usepackage{amsmath}  %% Mathematics
\usepackage{amsthm}
\usepackage{amsfonts}
\usepackage{url}      %% Hyperlinks
\usepackage{markdown} %% Lightweight markup
\usepackage{listings} %% Source code highlighting
\lstset{
  basicstyle      = \ttfamily,%
  identifierstyle = \color{black},%
  keywordstyle    = \color{blue},%
  keywordstyle    = {[2]\color{cyan}},%
  keywordstyle    = {[3]\color{olive}},%
  stringstyle     = \color{teal},%
  commentstyle    = \itshape\color{magenta}}
\usepackage{floatrow} %% Putting captions above tables
\floatsetup[table]{capposition=top}
\begin{document}
\chapter*{Introduction}
\addcontentsline{toc}{chapter}{Introduction}

Theses are rumoured to be the capstones of education, so I decided
to write one of my own. If all goes well, I will soon have a
diploma under my belt. Wish me luck!

\chapter{Background and research}
\section{Legato system}
\section{Similar tools}
\section{State-of-the-art of the legal documents processing}
\chapter{Natural Language Processing: Feature extraction}
The amount of text in the legal documents can vary largely.
Most of the documents we are working with stretch across one page, however, the average number of pages is 6 and we can find documents up to hundred pages long.
The typical approach of the text categorization techniques is to create a document-term matrix containing counts of terms in each document.
The problem with this approach is that the long documents are represented by a vector, which \emph{information gain} is rather low.
For us, a high \emph{information gain} means that the representation captures the general idea, purpose of the document, and similarity/dissimilarity with other documents.

To address the issue, we decided to represent a document by specific features extracted by one of the NLP tools.
We are looking for potentially the most interesting information in the text, such as people, locations, dates, crime related words or email addresses.
Nowadays, one can find numerous services providing extraction of many different features.
We have chosen some of them and have collected results from testing on our mock legal case.
We evaluated the results based on analysis of our use-case and on a survey, in which participants labeled features from the text in a similar way as a NLP tool.

\section{Introduction of the tools}

\subsection{Goog}

\section{Comparison}
\subsection{Mock legal case}
\section{Utilization}
\chapter{Classification}
\section{Attributes}
\section{Models}
\section{Parameter tuning and model selection}
\chapter{Association rule mining}
\section{Players}
\section{Locations}
\section{Dates}
\section{Key phrases}
\section{Contradiction detection}
\chapter{Implementation}
\section{Frameworks and libraries}
\section{Design}
\section{Integration into Legato}
\chapter{Evaluation}
\section{Dataset introduction}
\section{Results}
\chapter{Conclusion}



  \printbibliography[heading=bibintoc] %% Print the bibliography.

  \makeatletter\thesis@blocks@clear\makeatother
  \phantomsection %% Print the index and insert it into the
  \addcontentsline{toc}{chapter}{\indexname} %% table of contents.
  \printindex

\appendix %% Start the appendices.
\chapter{An appendix}
Here you can insert the appendices of your thesis.

\end{document}
